% Options for packages loaded elsewhere
\PassOptionsToPackage{unicode}{hyperref}
\PassOptionsToPackage{hyphens}{url}
%
\documentclass[
]{article}
\usepackage{lmodern}
\usepackage{amssymb,amsmath}
\usepackage{ifxetex,ifluatex}
\ifnum 0\ifxetex 1\fi\ifluatex 1\fi=0 % if pdftex
  \usepackage[T1]{fontenc}
  \usepackage[utf8]{inputenc}
  \usepackage{textcomp} % provide euro and other symbols
\else % if luatex or xetex
  \usepackage{unicode-math}
  \defaultfontfeatures{Scale=MatchLowercase}
  \defaultfontfeatures[\rmfamily]{Ligatures=TeX,Scale=1}
\fi
% Use upquote if available, for straight quotes in verbatim environments
\IfFileExists{upquote.sty}{\usepackage{upquote}}{}
\IfFileExists{microtype.sty}{% use microtype if available
  \usepackage[]{microtype}
  \UseMicrotypeSet[protrusion]{basicmath} % disable protrusion for tt fonts
}{}
\makeatletter
\@ifundefined{KOMAClassName}{% if non-KOMA class
  \IfFileExists{parskip.sty}{%
    \usepackage{parskip}
  }{% else
    \setlength{\parindent}{0pt}
    \setlength{\parskip}{6pt plus 2pt minus 1pt}}
}{% if KOMA class
  \KOMAoptions{parskip=half}}
\makeatother
\usepackage{xcolor}
\IfFileExists{xurl.sty}{\usepackage{xurl}}{} % add URL line breaks if available
\IfFileExists{bookmark.sty}{\usepackage{bookmark}}{\usepackage{hyperref}}
\hypersetup{
  pdftitle={Final Project},
  hidelinks,
  pdfcreator={LaTeX via pandoc}}
\urlstyle{same} % disable monospaced font for URLs
\usepackage[margin=1in]{geometry}
\usepackage{color}
\usepackage{fancyvrb}
\newcommand{\VerbBar}{|}
\newcommand{\VERB}{\Verb[commandchars=\\\{\}]}
\DefineVerbatimEnvironment{Highlighting}{Verbatim}{commandchars=\\\{\}}
% Add ',fontsize=\small' for more characters per line
\usepackage{framed}
\definecolor{shadecolor}{RGB}{248,248,248}
\newenvironment{Shaded}{\begin{snugshade}}{\end{snugshade}}
\newcommand{\AlertTok}[1]{\textcolor[rgb]{0.94,0.16,0.16}{#1}}
\newcommand{\AnnotationTok}[1]{\textcolor[rgb]{0.56,0.35,0.01}{\textbf{\textit{#1}}}}
\newcommand{\AttributeTok}[1]{\textcolor[rgb]{0.77,0.63,0.00}{#1}}
\newcommand{\BaseNTok}[1]{\textcolor[rgb]{0.00,0.00,0.81}{#1}}
\newcommand{\BuiltInTok}[1]{#1}
\newcommand{\CharTok}[1]{\textcolor[rgb]{0.31,0.60,0.02}{#1}}
\newcommand{\CommentTok}[1]{\textcolor[rgb]{0.56,0.35,0.01}{\textit{#1}}}
\newcommand{\CommentVarTok}[1]{\textcolor[rgb]{0.56,0.35,0.01}{\textbf{\textit{#1}}}}
\newcommand{\ConstantTok}[1]{\textcolor[rgb]{0.00,0.00,0.00}{#1}}
\newcommand{\ControlFlowTok}[1]{\textcolor[rgb]{0.13,0.29,0.53}{\textbf{#1}}}
\newcommand{\DataTypeTok}[1]{\textcolor[rgb]{0.13,0.29,0.53}{#1}}
\newcommand{\DecValTok}[1]{\textcolor[rgb]{0.00,0.00,0.81}{#1}}
\newcommand{\DocumentationTok}[1]{\textcolor[rgb]{0.56,0.35,0.01}{\textbf{\textit{#1}}}}
\newcommand{\ErrorTok}[1]{\textcolor[rgb]{0.64,0.00,0.00}{\textbf{#1}}}
\newcommand{\ExtensionTok}[1]{#1}
\newcommand{\FloatTok}[1]{\textcolor[rgb]{0.00,0.00,0.81}{#1}}
\newcommand{\FunctionTok}[1]{\textcolor[rgb]{0.00,0.00,0.00}{#1}}
\newcommand{\ImportTok}[1]{#1}
\newcommand{\InformationTok}[1]{\textcolor[rgb]{0.56,0.35,0.01}{\textbf{\textit{#1}}}}
\newcommand{\KeywordTok}[1]{\textcolor[rgb]{0.13,0.29,0.53}{\textbf{#1}}}
\newcommand{\NormalTok}[1]{#1}
\newcommand{\OperatorTok}[1]{\textcolor[rgb]{0.81,0.36,0.00}{\textbf{#1}}}
\newcommand{\OtherTok}[1]{\textcolor[rgb]{0.56,0.35,0.01}{#1}}
\newcommand{\PreprocessorTok}[1]{\textcolor[rgb]{0.56,0.35,0.01}{\textit{#1}}}
\newcommand{\RegionMarkerTok}[1]{#1}
\newcommand{\SpecialCharTok}[1]{\textcolor[rgb]{0.00,0.00,0.00}{#1}}
\newcommand{\SpecialStringTok}[1]{\textcolor[rgb]{0.31,0.60,0.02}{#1}}
\newcommand{\StringTok}[1]{\textcolor[rgb]{0.31,0.60,0.02}{#1}}
\newcommand{\VariableTok}[1]{\textcolor[rgb]{0.00,0.00,0.00}{#1}}
\newcommand{\VerbatimStringTok}[1]{\textcolor[rgb]{0.31,0.60,0.02}{#1}}
\newcommand{\WarningTok}[1]{\textcolor[rgb]{0.56,0.35,0.01}{\textbf{\textit{#1}}}}
\usepackage{graphicx}
\makeatletter
\def\maxwidth{\ifdim\Gin@nat@width>\linewidth\linewidth\else\Gin@nat@width\fi}
\def\maxheight{\ifdim\Gin@nat@height>\textheight\textheight\else\Gin@nat@height\fi}
\makeatother
% Scale images if necessary, so that they will not overflow the page
% margins by default, and it is still possible to overwrite the defaults
% using explicit options in \includegraphics[width, height, ...]{}
\setkeys{Gin}{width=\maxwidth,height=\maxheight,keepaspectratio}
% Set default figure placement to htbp
\makeatletter
\def\fps@figure{htbp}
\makeatother
\setlength{\emergencystretch}{3em} % prevent overfull lines
\providecommand{\tightlist}{%
  \setlength{\itemsep}{0pt}\setlength{\parskip}{0pt}}
\setcounter{secnumdepth}{-\maxdimen} % remove section numbering
\ifluatex
  \usepackage{selnolig}  % disable illegal ligatures
\fi

\title{Final Project}
\author{}
\date{\vspace{-2.5em}}

\begin{document}
\maketitle

\textbf{Finance 5350: Computational Financial Modeling}

\textbf{Due Date:} December 18, 2020 at Midnight

\hypertarget{preliminary}{%
\subsection{\texorpdfstring{\textbf{Preliminary}}{Preliminary}}\label{preliminary}}

Create a Python module in a file names \texttt{options.py}. A sample
file is included in the folder for this project. You will add to this
module as you progress through the final project. You will import it
into various notebooks that will contain your answers to the problems
below.

\hypertarget{part-i-the-european-binomial-option-pricing-model}{%
\subsection{\texorpdfstring{\textbf{Part I: The European Binomial Option
Pricing
Model}}{Part I: The European Binomial Option Pricing Model}}\label{part-i-the-european-binomial-option-pricing-model}}

Write a Jupyter notebook named \texttt{Part-One.ipynb} to solve the
following problems. You should import the \texttt{options.py} module in
the first code cell of the notebook as follows:

\begin{verbatim}
import options as opt
\end{verbatim}

\hypertarget{problem-1}{%
\subsubsection{\texorpdfstring{\textbf{Problem
1}}{Problem 1}}\label{problem-1}}

\begin{itemize}
\tightlist
\item
  \textbf{(a)} Complete the function \texttt{european\_binomial\_pricer}
  in the \texttt{options.py} module to implement the multiperiod
  European binomial option pricing model. This step is to be completed
  before you import the module into your notebook.
\end{itemize}

\begin{itemize}
\tightlist
\item
  \textbf{(b)} Verify that it works for both call and put options with
  \(n = 1\) (i.e.~a single period). Compare against a hand-written
  solution. Assume the following:

  \begin{itemize}
  \tightlist
  \item
    Let \(S_{0} = \$100\)
  \item
    Let \(K = \$105\)
  \item
    Let \(r = 8\%\)
  \item
    Let \(T = 1\) year
  \item
    Let \(\delta = 0.0\) (i.e.~no dividends)
  \item
    Let \(\sigma = 20\%\)
  \end{itemize}
\end{itemize}

\begin{itemize}
\tightlist
\item
  \textbf{(c)} Verify that it works for both call and put options with
  \(n = 3\). Compare against a hand-written solution. Use the same
  parameters as above in \textbf{(b)}.
\end{itemize}

\begin{itemize}
\tightlist
\item
  \textbf{(d)} What happens if you set \(n = 200\)? Solve for both the
  call and put prices. \textbf{DO NOT} try to solve by hand! Again, use
  the parameter values from \textbf{(b)}.
\end{itemize}

\hypertarget{problem-2}{%
\subsubsection{\texorpdfstring{\textbf{Problem
2}}{Problem 2}}\label{problem-2}}

\begin{itemize}
\tightlist
\item
  \textbf{(a)} Use the functions included in \texttt{options.py} to
  price the call and put option from \textbf{Problem 1} part
  \textbf{(b)} with the Black-Scholes option pricing model. See McDonald
  Chapter 12 for background on the Black-Scholes option pricing model.
\end{itemize}

\begin{itemize}
\tightlist
\item
  \textbf{(b)} Use the \texttt{european\_binomial\_pricer} function with
  the following values: \(n = 20, 40, 60, 80, \ldots, 200\)
  (i.e.~increment by \(20\)). Compare to the Black-Scholes prices
  obtained above. Make a table to report the results. What can you say
  about the European Bimomial model relative to the Black-Scholes model?
  Discuss the convergence of the European Bimomial to the Black-Scholes
  model.
\end{itemize}

\hypertarget{part-ii-the-american-binomial-option-pricing-model}{%
\subsection{\texorpdfstring{\textbf{Part II: The American Binomial
Option Pricing
Model}}{Part II: The American Binomial Option Pricing Model}}\label{part-ii-the-american-binomial-option-pricing-model}}

Write a Jupyter notebook named \texttt{PartTwo.ipynb} to solve the
following problems. You should import the \texttt{options.py} module in
the first code cell of the notebook as follows:

\begin{verbatim}
import options as opt
\end{verbatim}

\hypertarget{problem-1-1}{%
\subsubsection{\texorpdfstring{\textbf{Problem
1}}{Problem 1}}\label{problem-1-1}}

\begin{itemize}
\tightlist
\item
  Using the functions \texttt{european\_binomial\_call} and
  \texttt{european\_binomial\_put} as starting points, implement the
  functions \texttt{american\_binomial\_call} and
  \texttt{american\_binomial\_put}. These functions should solve the
  optimal stopping problem implicit in the American option pricing
  problem. Write your solutions in the \texttt{options.py} module. This
  step is to be completed before you import the module for the problems
  below.
\end{itemize}

\hypertarget{problem-2-1}{%
\subsubsection{\texorpdfstring{\textbf{Problem
2}}{Problem 2}}\label{problem-2-1}}

\begin{itemize}
\tightlist
\item
  \textbf{Set-up:} Let \(S_{0} = \$100\), \(K = \$95\), \(r = 8\%\)
  (continuously compounded), \(\sigma = 30\%\), \(\delta = 0\),
  \(T = 1\) year, and \(n = 3\).
\end{itemize}

\begin{itemize}
\tightlist
\item
  \textbf{(a)} Verify that the binomial option price for an American
  call option is \(\$18.283\). Verify that there is never early
  exercise; hence a European call would have the same price. Compare
  your Python solution to a hand-written solution.
\end{itemize}

\begin{itemize}
\tightlist
\item
  \textbf{(b)} Show that the binomial option price for a European put
  option is \(\$5.979\). Verify that put-call parity is satisfied.
\end{itemize}

\begin{itemize}
\tightlist
\item
  \textbf{(c)} Verify that the price of an American put is \(\$6.678\).
\end{itemize}

\begin{itemize}
\tightlist
\item
  \textbf{(d)} Repeat each of the above for \(n = 200\). How can you be
  sure there is never early exercise of the American call from part
  \textbf{(a)}? \textbf{DO NOT} attempt to solve this part by hand!
\end{itemize}

\hypertarget{problem-3}{%
\subsubsection{\texorpdfstring{\textbf{Problem
3}}{Problem 3}}\label{problem-3}}

\begin{itemize}
\tightlist
\item
  Repeat the previous problem assuming that the stock pays a continuous
  dividend of \(8\%\) per year (continuously compounded).
\item
  Calculate the prices of the American and European puts and calls.
\item
  Which options are early-exercised? Explain your answer.
\end{itemize}

\begin{Shaded}
\begin{Highlighting}[]

\end{Highlighting}
\end{Shaded}

\hypertarget{part-iii-simulating-binomial-trees}{%
\subsection{\texorpdfstring{\textbf{Part III: Simulating Binomial
Trees}}{Part III: Simulating Binomial Trees}}\label{part-iii-simulating-binomial-trees}}

Write a Jupyter notebook named \texttt{PartThree.ipynb} to solve the
following problems. Make sure to import the \texttt{options.py} module
at the top of your notebook. You should know how to do that by now, so I
won't belabor the point.

\hypertarget{problem-1-2}{%
\subsubsection{\texorpdfstring{\textbf{Problem
1}}{Problem 1}}\label{problem-1-2}}

\begin{itemize}
\tightlist
\item
  Complete the function \texttt{binomial\_path} in the
  \texttt{options.py} module that simulates a binomial path.
\item
  This step is to be completed prior to being imported for the problem
  below.
\end{itemize}

\hypertarget{problem-2-2}{%
\subsubsection{\texorpdfstring{\textbf{Problem
2}}{Problem 2}}\label{problem-2-2}}

\begin{itemize}
\tightlist
\item
  \textbf{Set-up:} Let \(S_{0} = \$100\), \(r = 8\%\) (continuously
  compounded), \(\sigma = 30\%\), \(\delta = 5\%\), and \(T = 1\) year.
\item
  Set \(n = 252\) (i.e.~roughly the number of trading days per year so
  that each sub-period is a single day).
\item
  Simulate a binomial path using your new function.
\item
  Use the \texttt{Matplotlib.pyplot} function \texttt{plot} to make a
  plot of your simulated path. Label your axes appropriately.
\end{itemize}

\end{document}
